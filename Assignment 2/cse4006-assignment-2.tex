\documentclass[conference]{IEEEtran}
% \IEEEoverridecommandlockouts
% The preceding line is only needed to identify funding in the first footnote. If that is unneeded, please comment it out.
\usepackage{cite}
\usepackage{amsmath,amssymb,amsfonts}
\usepackage{algorithmic}
\usepackage{graphicx}
\usepackage{textcomp}
\usepackage{xcolor}
\usepackage{hyperref}
\usepackage{subfig}
\usepackage{subcaption}
\usepackage{float}
\def\BibTeX{{\rm B\kern-.05em{\sc i\kern-.025em b}\kern-.08em
    T\kern-.1667em\lower.7ex\hbox{E}\kern-.125emX}}
\begin{document}

\title{Habitat Connect\\
% We comment the sub-titles and funding for now
% {\footnotesize \textsuperscript{*}Note: Sub-titles are not captured in Xplore and should not be used}
% \thanks{Identify applicable funding agency here. If none, delete this.}
}


\author{\IEEEauthorblockN{\textsuperscript{} Laura Rodríguez López}
\IEEEauthorblockA{\textit{Department of Computer Science} \\
\textit{Hanyang University}\\
Madrid, Spain \\
1postlaura@gmail.com}
\and
\IEEEauthorblockN{\textsuperscript{} Christian Alexander Hernandez}
\IEEEauthorblockA{\textit{Department of Computer Science} \\
\textit{Hanyang University}\\
Dallas, United States \\
ch407541@gmail.com}
\and
\IEEEauthorblockN{\textsuperscript{} Marcel Plesky}
\IEEEauthorblockA{\textit{Department of Information Systems} \\
\textit{Hanyang University}\\
Zürich, Switzerland \\
marcel.plesky68@gmail.com}
}
\maketitle


% ABSTRACT
\begin{abstract}
Habitat Connect is a web application designed for
residents of Hanyang Habitat, our apartment complex, to stay
connected, regardless of their cultural backgrounds. Our
platform offers features like an announcement board, event
calendar, washer checker, and maintenance request page.
Residents can easily receive important updates, plan events,
check laundry availability, and report maintenance concerns –
all within a user-friendly interface. With Habitat Connect, we
hope to foster a stronger sense of community and make
everyday tasks more convenient for everyone in Hanyang
Habitat.
\end{abstract}


% No keywords for know
%\begin{IEEEkeywords}
%component, formatting, style, styling, insert
%\end{IEEEkeywords}


% ROLE ASSIGNMENTS + TABLE
\section*{Role Assignments}
In the table below, we can find a detailed breakdown of the roles and responsibilities assigned to each team member throughout the duration of the project.
\begin{table}[htbp]
\centering
\begin{tabular}{|p{1.6cm}|l|p{3.6cm}|}
\hline
\textbf{Role} & \textbf{Name} & \textbf{Task Description}\\
\hline
User/Customer & Laura & 
- Uses the software \\
& & 
- Establishes the conditions and features that the software must satisfy (requirements) \\
& & 
- Provides feedback \\
& & 
- Approves change requests\\
\hline
Software Developer & Christian & 
- Works with user/customer to gather their needs \\
& & 
- Designs and develops the software \\
& & 
- Tests and debugs the software\\
\hline
Development Manager & Marcel & 
- Supervises the software development process \\
& & 
- Facilitates communication between user/customer and software developer \\
& & 
- Manages tools and resources \\
& & 
- Monitors progress and quality of the software\\
\hline
\end{tabular}
\caption{Roles and Responsibilities}
\label{tab1}
\end{table}


% INTRODUCTION
\section{Introduction}
\subsection{Motivation}
When international students arrive at Hanyang University, they often seek affordable and short-term housing options. One of the most popular accommodations are Goshiwons like Hanyang Habitat. As members of this team currently living in Habitat, we would like to provide residents with a new toll that caters to their needs.

Living in Hanyang Habitat has given us firsthand insight into the challenges faced by exchange students. With approximately 70 international residents that come from more than 15 countries, it's evident that online communication platforms vary widely. For instance, Europeans prefer applications like WhatsApp or Instagram, while North Americans rely heavily on text messaging. In contrast, WeChat is very popular in certain Asian countries like China or Indonesia. In South Korea, KakaoTalk is the go-to messaging app, but its unique features and interface can pose a learning curve for newcomers, especially for those that do not speak Korean.


\subsection{Problem Statement}
Given these cultural differences regarding communication tools, we asked ourselves: which platform should Hanyang Habitat residents use to effectively communicate with one another? Our proposed solution is Habitat Connect, an announcement board with a user-friendly English interface.

As well as serving as an online communication tool, Habitat Connect offers more features that focus on the needs of Habitat residents. This includes a laundry checker that provides real-time updates on washer availability, an event and birthday calendar to keep everyone informed about upcoming celebrations, and an easy-to-use maintenance request system, ensuring a direct line of communication between the tenants and the apartment manager.

Habitat Connect ensures that all residents, regardless of their language and online backgrounds, can comfortably navigate and utilize the tool. This idea comes from a desire to foster a sense of community among students in Hanyang Habitat. We aim to enhance the living experience of international students, making their time at Hanyang University more comfortable and rewarding.  


\subsection{Research on related software}
\section*{WhatsApp}
WhatsApp is a widely used messaging application that offers instant messaging, voice and video calls, file-sharing and even a poll-system that allows voting for user given options. WhatsApp is used in family, friends and even work environments to stay connected and up to date with your contacts.
\section*{KakaoTalk}
Just like WhatsApp, KakaoTalk is a popular messaging application. It offers even more features than WhatsApp like mobile payment and other various app services making it a multifunctional communication platform.
\section*{Discord}
Discord is known for its features like text and voice chat, creating and managing channels, and building communities around specific interests or topics. It's often used for both casual and structured discussions, making it a versatile option for promoting communication and collaboration in a variety of settings, including apartment buildings and shared living spaces.
\section*{Google Calendar}
Google Calendar is a user-friendly online calendar application from Google. It allows users to create, manage and share events, streamlining planning. Moreover, users can also import birthdays from their contacts, making it easier to remember important dates. Google integration and various viewing options make it a valuable time management and organization tool.
\section*{FamilyWall}
FamilyWall is an application designed to help families and individuals organize their lives, share information, and manage family-related activities. It allows users to create shared calendars, to-do lists, shopping lists, and share photos, as well as important information with family members or within their respective groups.
\section*{Laundry Applications}
Some laundromats like Samsung and LG have developed their own mobile apps to check the availability of washing machines and dryers.
\section*{Zendesk}
Zendesk is a popular ticket management system used for customer support and issue tracking. It helps organize and track customer inquiries and support tickets. With Zendesk, businesses can provide better customer service.


% REQUIREMENT ANALYSIS
\section{Requirement Analysis}
\subsection{Sign up/Log in}
If users are not registered on the database, they must sign up with an email address and a password. Moreover, to complete the registration process, it is necessary to provide a username, birthdate, and Habitat’s room number.
Registered users can log in with their username and password.
\subsection{Recent announcements}
Recent Announcements is the main page. The user is able to see all announcements ever posted, except those deleted by their corresponding authors. Regarding announcements, the following actions are available for the user:
\begin{itemize}
    \item Post an announcement: user can write text and insert images in an announcement. Once they choose to post the announcement, it will be available in the Recent announcements page to every user registered in Habitat Connect. The ownership of the announcement is registered as well.
    \item Edit an announcement: users can edit their posted announcements.
    \item Delete an announcement: users can delete their posted announcements.
    \item Comment on an announcement: registered users can comment on posted announcements. Comments can only contain ASCII characters. Furthermore, the ownership of the comment is available for every user to see.
\end{itemize}
\subsection{Calendar view}
The system will provide a calendar view for users that automatically displays other users' birthdays based on their sign-up information. The calendar view will be monthly, and it can be switched between months with the help of navigation arrows on each side of the calendar. Users should be able to click on a specific date on the calendar to view the list of birthdays for that date, including the names of the users celebrating their birthdays.
\subsection{Laundry checker}
This function will be on the left side of the main page in the form of an icon, and once clicked, users will be taken to another page where they will see each washer, which is assigned a unique number. The status of each washer is displayed in the Status column, with different colors to make it visually clear: IN USE is displayed in red, AVAILABLE is displayed in green, and FINISHED is displayed in white. Visually, it will be in a table-like format for easy understanding. Each washer has a set of action buttons next to it. The available actions are:
\begin{itemize}
    \item AVAILABLE (green button): users can click this button when they are going to use the washer. Once clicked, the washer’s status will change to IN USE, in the color red.
    \item FINISHED (white button): once a user is done using the washer, they will click a button that will change the status back to AVAILABLE, displayed in green.
\end{itemize}

There may be additional features such as timers to display how long a washer has been in use, as well as user-friendly tooltips or information icons that can provide additional information or instructions.
\subsection{Maintenance requests}
Under the laundry checker icon, we can find the Maintenance Request icon. Once it is clicked, users will be taken to a new page. There will be two text boxes to create a Ticket: the first textbox will be the ticket name (for example “No Hot Water”), and the second text box will be the description of the maintenance issue with a maximum of 200 characters. Once the ticket is set up, there will be a Submit Ticket button at the bottom of the page. Once clicked, an email will automatically be sent to the apartment manager's private email. The page for the user after clicking the Submit Ticket button will also change. In the first half of the page, we will find a message saying “Your ticket has been submitted!”. Under that will be the status of the ticket which can be one of two options: Under Review (which will automatically be selected), and Ticket Completed. Once the user’s request has been completed in person, they can click the Ticket Completed option on the Maintenance Request page, which will update the user’s page back to the original create a ticket page.
\subsection{Loading page}
Upon opening our website, a 3 second screen with just our logo will appear on the screen. This is the general convention for websites, we can see an example of this in X or discord. After 3 seconds, the page will load into the recent announcements page. Moreover, the implementation of this loading screen is not merely a matter of aesthetic preference; it serves a functional purpose as well. The brief delay allows essential elements of the website to load efficiently, resulting in a smoother and more responsive browsing experience for our users. This loading screen will only be seen upon reloading the site, not when switching between the different functions. The reason for this is for the user's convenience, as they won’t need to wait for the page to load every time they want to use a new function.
\subsection{Notifications}
Users will only receive notifications regarding the status of their maintenance tickets, and the events that are occurring the day of. They can see their notifications on the top right of your screen, where the profile is located. Next to the profile, there will be a bell icon with a hand number, signifying how many notifications the user has received. Once the bell is clicked, users will see their notifications which again, can either be the ticket status or the events that are occurring that day. Users can click the Clear button to remove the notifications. However, once done, they cannot go back to see what was there. This will be the extent of the notifications, as we do not want to overflow it with other unnecessary added notifications.
\subsection{User feedback Support}
In order to manage the website without bugs, we need a substantial amount of feedback from users. This is a function that most every high level website has in order to enhance user experience. At the top right of the screen, where “Profile” is, one of the options under above “Settings” is a “Report Bug” option. Once clicked, the user will be taken to a separate page, where similar to an email, will have a Title and description text box with a maximum of 300 characters. Once both are filled and the user clicks “Submit”, an email will automatically be sent to an admin regarding the bug. This will also be used as a “Support Ticket”, whenever someone needs to get in contact with an admin regarding an issue they have about their account.
\subsection{User profile options}
The user profile is a fundamental component of an application, serving as a repository of user-specific information and a platform for users to manage and personalize their experience within the system. The user profile typically encompasses several key options and functionalities designed to cater to the diverse needs and preferences of the application's users. However, to keep the concept of this web application, the changes the user can make to their profile is limited. On the “Profile” section and “Edit” subsection, the user will be taken to a new page, where they can apply one of 3 options regarding their account. The first option is to switch to anonymous mode, where their room number and name will be hidden. With this, the user is still able to read the announcement chat and use the washer checker, however they cannot type in the announcement chat or create a maintenance manager ticket. The second option is to edit their username, as upon signing up for our website, their real names are used for the users accounts. The users only have the option to change their usernames once a month. The last option they have is to change their profile pictures. There is no limit to this option and the users can change it as many times as they please.


% DEVELOPMENT ENVIRONMENT
\section{Development environment}
\subsection{Choice of software development platform}
Given that Habitat Connect is a web application, the choice of operating system is quite flexible. That is, we can choose any operating system that supports web development tools. To be more specific, our working environment consists of the following:
\begin{table}[htbp]
\centering
\begin{tabular}{|p{1.6cm}|l|p{3.6cm}|}
\hline
\textbf{Name} & \textbf{Version} & \textbf{Description}\\
\hline
Windows & 11 Home & 
- User-friendly environment with an updated UI \\
& & 
- Wide range of web development tools and IDEs \\
\hline
macOS & 12.5 Apple M1 Pro & 
- Unix-based \\
& & 
- Offers a powerful terminal for web development \\


\hline
\end{tabular}
\caption{Working environment}
\label{tab2}
\end{table}

Moreover, our development process will make use of some tools that can be found below.
\begin{enumerate}

    \item \textsc{Svelte} is a JavaScript library that streamlines the process of building user interfaces. It compiles components directly into efficient, imperative code. This leads to faster rendering and a smoother user experience.
    \begin{figure}[H]
    \centering
    \includegraphics[width=3cm]{figs/svelte_logo.png}
    \caption{Svelte logo}
    \label{fig:Svelte logo}
    \end{figure}

    \item \textsc{Node.js} is a server-side runtime environment for executing JavaScript code, enabling efficient web application development. It excels in handling real-time interactions and concurrent requests
    \begin{figure}[H]
    \centering
    \includegraphics[width=3cm]{figs/nodejs_logo.png}
    \caption{Node.js logo}
    \label{fig:Node.js logo}
    \end{figure}
    
    \item \textsc{Bootstrap} is a front-end framework that provides pre-designed, responsive, and customizable UI components. It allows us to create appealing and web applications with ease.
    \begin{figure}[H]
    \centering
    \includegraphics[width=3cm]{figs/Bootstrap_logo.png}
    \caption{Bootstrap logo}
    \label{fig:Bootstrap logo}
    \end{figure}
    
    \item \textsc{Axios} is a promise-based HTTP client for the browser and Node.js. It simplifies making HTTP requests, handling responses, and managing asynchronous tasks in JavaScript. We will mainly use it for fetching data from APIs.
    \begin{figure}[H]
    \centering
    \includegraphics[width=3cm]{figs/Axios_logo.png}
    \caption{Axios logo}
    \label{fig:Axios logo}
    \end{figure}
    
    \item \textsc{MongoDB} is a NoSQL database that provides a flexible, document-based data structure. It's designed for handling large amounts of unstructured data, making it suitable for dynamic web applications.
    \begin{figure}[H]
    \centering
    \includegraphics[width=3cm]{figs/MongoDB_logo.png}
    \caption{MongoDB logo}
    \label{fig:MongoDB logo}
    \end{figure}

    \item \textsc{Visual Studio Code} is a powerful code editor that supports various programming languages. Furthermore, this IDE offers features like syntax highlighting, debugging, extensions, and Git integration.
    \begin{figure}[H]
    \centering
    \includegraphics[width=3cm]{figs/visualstudio_code_logo.png}
    \caption{Visual Studio Code logo}
    \label{fig:Visual Studio Code logo}
    \end{figure}

    \item \textsc{Git/GitHub}: Git is a version control system that allows multiple developers to collaborate on a project. On the other hand, GitHub is a web-based platform that provides Git repositories, collaboration tools, and hosting services. It facilitates efficient team collaboration and code management.
    \begin{figure}[H]
    \centering
    \subfloat[Git logo]{\includegraphics[width=3cm]{figs/git_logo.png}}
    \subfloat[GitHub logo]{\includegraphics[width=3cm]{figs/GitHub_logo.png}}
    \caption{Logos for Git and GitHub}
    \label{fig:Logos}
    \end{figure}    
    
    \item \textsc{Google Drive} is a cloud-based storage and collaboration platform. It allows us to store files, share documents, and collaborate in real-time. It's useful for storing mock documents that will be pushed to GitHub.
    \begin{figure}[H]
    \centering
    \includegraphics[width=3cm]{figs/google_drive_logo.png}
    \caption{Google Drive logo}
    \label{fig:Google Drive logo}
    \end{figure}
    
    \item \textsc{Overleaf} is an online platform for collaborative document editing. It simplifies the process of creating and editing scientific and technical documents, so it’s very popular among researchers and academics. We will make use of it to create our LaTeX documents.  
    \begin{figure}[H]
    \centering
    \includegraphics[width=3cm]{figs/overleaf_logo.png}
    \caption{Overleaf logo}
    \label{fig:overleaf logo}
    \end{figure}

    \item \textsc{Moqups} is a web-based design and prototyping tool that allows users to collaborate on UI design in real-time
    \begin{figure}[H]
    \centering
    \includegraphics[width=6cm]{figs/moqups_logo.png}
    \caption{Moqups logo}
    \label{fig:Moqups logo}
    \end{figure}
    
\end{enumerate}

\vspace{12pt}
Regarding the choice of programming language, we will be using the following:
\begin{enumerate}
    \item \textsc{JavaScript} is a high-level programming language primarily known for its use in web development. It allows us to add interactivity and dynamic behavior to web pages. With Node.js, it can also be used for server-side development.
    \begin{figure}[H]
    \centering
    \includegraphics[width=3cm]{figs/JavaScript-Logo.png}
    \caption{JavaScript logo}
    \label{fig:JavaScript logo}
    \end{figure}
    
    \item \textsc{html} is the standard language used for creating web pages. It provides the structure and content of a webpage, using elements like headings, paragraphs, and links.
    \begin{figure}[H]
    \centering
    \includegraphics[width=3cm]{figs/html_logo.png}
    \caption{HTML logo}
    \label{fig:HTML logo}
    \end{figure}
    
    \item \textsc{css} is a language used for styling web pages. It controls the visual presentation of HTML elements, including aspects like layout, colors, fonts, and animations.
    \begin{figure}[H]
    \centering
    \includegraphics[width=3cm]{figs/css_logo.png}
    \caption{CSS logo}
    \label{fig:CSS logo}
    \end{figure}
    
    \item \textsc{Python} is a user-friendly and versatile programming language known for its simplicity and readability.
    \begin{figure}[H]
    \centering
    \includegraphics[width=3cm]{figs/python_logo.png}
    \caption{Python logo}
    \label{fig:Python logo}
    \end{figure}
    
\end{enumerate}

All software development tools are free of use. Furthermore, since Windows11 and macOS are already in use, we will not count them as costs.

\subsection{Software in use}
Today people rely on widely-used software like Discord for chatting, Google Calendar for planning, and FamilyWall for staying organized. Our upcoming project takes inspiration from these popular tools and blends them into one. Our software will make it easy for people living in the same place to chat, schedule events, and share information, enhancing the way they interact and stay connected.

\subsection{Task distribution}
The responsibilities of each member are yet to be determined. However, we are thinking of splitting the workload in the following categories:
\begin{itemize}
    \item Management: responsible for checking up on each team member work and coordinating the overall work to meet due dates
    \item Documentation: responsible for writing the pdf and creating the LaTex files
    \item Front-end: responsible for developing the front-end side of the software
    \item Back-end: responsible for developing the back-end side of the software
\end{itemize}
Our idea is that each team member will choose which skills they want to improve best, as long as the work is fairly divided. We believe that not dividing the tasks will result in an inefficient and not realistic way of software development.


% SPECIFICATIONS
\section{Specifications}
In the section below, we will provide a detailed description of the implementation process of each requirement.


\subsection{Sign up/Log in}
If logged out, or a user is entering the website for the first time, the first component that users will see is the login page, as the website is a private application not open for public use. This page will have two main functions to enter the main webpage, sign up and login. The user sign up process on Habitat Connect is the starting point for users to establish their digital presence. It allows them to create unique usernames, set up secure passwords, provide email addresses, and specify room numbers. These steps collectively enable users to register and create a personalized digital identity within the platform, enhancing their overall user experience.
    \begin{figure}[H]
    \centering
    \includegraphics[width=9cm]{figs/log_in.png}
    \caption{Log in page}
    \label{fig:Log in page}
    \end{figure}
    
\section*{Log In}
The user login process in Habitat Connect is the secure gateway to the platform, offering privacy and security for the users.
\subsection*{Correct login}
If the correct information is inputted, the page will load to our personalized loading screen into the main Announcement Board page.
\subsection*{Incorrect password}
Users will have a maximum of 5 password attempts. If exceeded, the user will be locked out for an hour. When an invalid password is entered, a message in red will popup. If the user forgets their password, they can click the “Forgot password” button where they will receive email notification associated with the account which will redirect the users to a page to create a new password.
    \begin{figure}[H]
    \centering
    \includegraphics[width=9cm]{figs/incorrect_password.png}
    \caption{Incorrect password popup}
    \label{fig:Incorrect password popup}
    \end{figure}
\section*{Sign up}
Once the sign up  button at the top right of the page is clicked, the user will be taken to a new page to fill out the required fields in order to login. To ensure a robust and personalized user experience on Habitat Connect, certain fields stand as mandatory during the sign-up process. These include room number, a valid email address, a password meeting specific security requirements, a unique username, and the user's date of birth. These essential details enable users to establish their digital identity, ensure secure access, and facilitate personalized interactions within the platform. Below, we can find the fields specification details:
\begin{itemize}
    \item \texttt{Username:} maximum string of 12 characters. Special characters are allowed
    \item \texttt{Password:} must exceed 8 characters. Special characters allowed and must included 1 capitalized characters
    \item \texttt{Email:} must provide valid email
    \item \texttt{Birth date:} MM/DD/YYYY order, automatically inducted to the birthday calendar once signed up
    \item \texttt{Room number:} maximum 4 integers corresponding to tenants room number. No characters allowed
\end{itemize}
    \begin{figure}[H]
    \centering
    \includegraphics[width=9cm]{figs/sign_up.png}
    \caption{Sign up form}
    \label{fig:Sign up form}
    \end{figure}


\subsection{Recent announcements}
The announcement board will be the first thing that users will see upon logging into our web application. This will be considered our “Main Page”. This screen will allow users to perform a specific number of actions.
    \begin{figure}[H]
    \centering
    \includegraphics[width=9cm]{figs/recent_announcements.png}
    \caption{Recent announcements page}
    \label{fig:Recent announcements page}
    \end{figure}
Habitat Connect offers a range of communication and interaction features for its users. These features include the ability to post announcements, which can include text, as well as making information accessible to all registered users and ensuring transparency through clear ownership attribution. Users can also edit their announcements to keep them up-to-date and relevant, and they have the option to delete announcements if they become outdated or if the announcement was simply canceled. Additionally, users can engage in discussions by commenting on announcements, with comment ownership clearly visible, promoting accountability and a sense of community within the platform.
    \begin{figure}[H]
    \centering
    \includegraphics[width=9cm]{figs/announcement_options.png}
    \caption{Announcement options}
    \label{fig:Announcement options}
    \end{figure}

\section*{Post Announcement button}
Once clicked, the user will be taken to a different screen, where they have to input 3 options: title, date, and location. All of these fields must be completed in order to post an announcement. 
\begin{itemize}
    \item \texttt{Title:} character string of maximum 50 characters
    \item \texttt{Date:} DD/MM/YYYY format
    \item \texttt{Location:} character string  of maximum 70 characters
\end{itemize}
Once the user clicks the “Post Announcement” button, the screen will reload into the main Announcement page with the new announcement updated.
    \begin{figure}[H]
    \centering
    \includegraphics[width=9cm]{figs/post_annnouncement.png}
    \caption{Post announcement page}
    \label{fig:Post announcement page}
    \end{figure}

\section*{Edit Announcement button}
This button will be visible directly on the announcement the user wants to edit. Once clicked, the user will be taken to another page, similar to the Post Announcement page where they can edit the title, date, or location. The only difference between this page and the Post Announcement page is the “Post Announcement” button being changed to “Save”. Once clicked, the updated content will be visible on Announcement.
    \begin{figure}[H]
    \centering
    \includegraphics[width=9cm]{figs/edit_announcement.png}
    \caption{Edit announcement page}
    \label{fig:Edit announcement page}
    \end{figure}

\section*{Delete Announcement button}
This button will be on the same row as the Edit Announcement Button. Once clicked, the user will be given a confirmation popup saying “Are you sure you want to delete this announcement?”, with a Cancel/Delete option. If delete is clicked,  the announcement will be deleted immediately without the page reloading, and the announcement below it will move up one.
    \begin{figure}[H]
    \centering
    \includegraphics[width=9cm]{figs/delete_announcement_popup.png}
    \caption{Delete announcement confirmation popup}
    \label{fig:Delete announcement confirmation popup}
    \end{figure}

\section*{Comment button}
This button will also be on the same row as Edit and Delete. Once clicked, the user will be taken to another page. In this page, the comment history of past users can be seen on the announcement, as well as a text box where users can type and post a comment, which will also be the only action available for the user in this page.  The order will be from the latest comment to last.
\begin{itemize}
    \item \texttt{Post comment text box:} character string maximum 150 characters
\end{itemize}
    \begin{figure}[H]
    \centering
    \includegraphics[width=9cm]{figs/comment_announcement_example.png}
    \caption{Comment example}
    \label{fig:Comment example}
    \end{figure}
Once the “Post comment” button is clicked, the comment will be at the top of the list of comments.


\subsection{Calendar view}
The calendar view can be directly accessed through the task bar on the left side of the screen. It is a dedicated feature within the Habitat Connect platform, designed to commemorate and celebrate the special occasions of its users. It enables users to keep track of birthdays, create memorable moments, and foster a sense of community through shared celebrations. This page will not have any actions, and is more of a visual aid for birthdays. Users will see the birthdays of the current month, but it is also possible to change the calendar view to display past/next months birthdays.
    \begin{figure}[H]
    \centering
    \includegraphics[width=9cm]{figs/calendar_view.png}
    \caption{Calendar view page}
    \label{fig:Calendar view page}
    \end{figure}
Upon clicking a birthday on the calendar, a popup will showcase, which allows users to see the detailed view of the birthdays.
    \begin{figure}[H]
    \centering
    \includegraphics{figs/birthday_details.png}
    \caption{Birthday details}
    \label{fig:Birthday details}
    \end{figure}


\subsection{Laundry checker}
Another one of the main four functions of the application is the laundry checker, which will have several actions that the user is allowed to do. Visually, it will consist of a table with three rows: washing machine number, current status, and change status. The change status row is the only row where users can perform an action.
    \begin{figure}[H]
    \centering
    \includegraphics[width=9cm]{figs/laundry_checker_page.png}
    \caption{Laundry checker page}
    \label{fig:Laundry checker page}
    \end{figure}

\section*{Current status}
There is no action to be done in this row. It is simply to show the status of the washer and is completely dependent on the “Change status” row. Moreover, if a user forgets to click the “Finish” button once their wash is finished, the status will automatically change to “Available” after 2 hours.

\section*{Change status}
The users can perform two actions, either click the “Finish” button or the “Use” button. If the status is “In Use”, users only have the option to click the “Finish” button. On the other hand, if the status is “Available”, the user will only have the option to click the “Use” button.
\begin{itemize}
    \item \texttt{Finish button:} only available when Status is “In use”. Once clicked, the status will automatically change to "Available".
    \item \texttt{Use button:} only available when status is “Available”. Once clicked, the status will automatically change to “In use”.
\end{itemize}


\subsection{Maintenance requests}
Habitat Connect simplifies the process of submitting maintenance requests, streamlining the user experience. Users can conveniently log requests for repairs, improvements, or assistance, ensuring that their concerns are addressed promptly. The platform not only fosters efficiency but also enhances transparency by allowing users to track the status and progress of their requests. Upon opening the maintenance requests tab on the taskbar, the user will be taken to a new page in which they can perform a maintenance request.
The requests made by the users contain the following fields:
\begin{itemize}
    \item \texttt{Title:} string of maximum 20 characters
    \item \texttt{Description:} string of maximum 300 characters
\end{itemize}
    \begin{figure}[H]
    \centering
    \includegraphics[width=9cm]{figs/submit_ticket_page.png}
    \caption{Submit ticket page}
    \label{fig:Submit ticket page}
    \end{figure}
Once the user clicks the “Submit Ticket” button, the page will update, confirming the user's maintenance request in the form of a ticket. A table will also be shown, showcasing the ticket ID, ticket status, and ticket title. No action can be performed once loaded to this page. Only one ticket is allowed at a time for a user. Furthermore, the ticket will automatically clear after three days.
    \begin{figure}[H]
    \centering
    \includegraphics[width=9cm]{figs/submitted_ticket_example.png}
    \caption{Example of a submitted ticket in progress}
    \label{fig:Example of a submitted ticket in progress}
    \end{figure}
The ticket, once submitted, will be automatically sent to an admin/apartment manager in the form of an email. At this point, the maintenance issue should be verbally deliberated between the tenant and admin. Once the maintenance request has been solved, it will be displayed as a completed ticket. 
    \begin{figure}[H]
    \centering
    \includegraphics[width=9cm]{figs/submitted_ticket_example_2.png}
    \caption{Example of a submitted ticket solved}
    \label{fig:Example of a submitted ticket solved}
    \end{figure}


\subsection{Loading page}
Upon the initial launch of the page, users can expect a brief loading time of approximately 3 seconds before reaching the main announcement board. This short delay allows the various elements and objects on the web page to load smoothly, ensuring a seamless and enjoyable browsing experience. The loading screen will consist of our logo and a loading icon upon every reload. Scrolling between taskbar functions will not cause the main reload visuals to pop up.
    \begin{figure}[H]
    \centering
    \includegraphics[width=9cm]{figs/loading_page.png}
    \caption{Loading page}
    \label{fig:Loading page}
    \end{figure}


\subsection{Notifications}
Notifications are an essential part of a community platform to make sure all users are caught up on announcements and their maintenance requests, as well as each other’s  birthdays. For users to check their notifications, they must look at the top right of their screen and locate a bell with an exclamation point. Once they click that bell, they will see a listing of their notifications. In order to keep a simplistic approach and not overwhelm users with unnecessary notifications, they will only receive alerts from new announcements, the status of maintenance requests, and birthdays. Moreover, one of the main causes for concern is if a announcement notification text is too long to fit in the bell’s textbox, however, to compensate, if a notification strings surpass the max notification string of  more than 13 characters, then on that 13th character, there will be a “...” to signify there is a longer message being transcribed. 
    \begin{figure}[H]
    \centering
    \includegraphics[width=9cm]{figs/notifications_example.png}
    \caption{Example of notifications}
    \label{fig:Example of notifications}
    \end{figure}


\section*{Clear button}
To clear every notification in the bell box, users can safely click the “Clear” button at the bottom middle of the bell box. Once clicked, the notification will be completely cleared and  cannot be retrieved again.

\section*{Exit button}
There is no actual exit button. Instead, to exit the notification bell box, users can click anywhere on the screen and it will dissipate the text box. 


\subsection{User feedback support}
Customer feedback is highly valued and plays a pivotal role in our commitment to delivering an exceptional user experience. Input from our valued users is incredibly important, as it enables us to continually enhance services. If users have suggestions for new features, come across any issues, or wish to share their thoughts, they should refer to the “User Feedback Support” button at the bottom left of the web page. Developers will be weary of every feedback, as the submit button will make a new entry of bugs or possible new features for the admins in the database collection. 
    \begin{figure}[H]
    \centering
    \includegraphics{figs/feedback_button.png}
    \caption{User feedback button}
    \label{fig:User feedback button}
    \end{figure}

\section*{User feedback button}
Upon clicking the “User Feedback” button, a text box will automatically appear. In this text box, a user can provide feedback or report bugs on the web page within 150 characters.
    \begin{figure}[H]
    \centering
    \includegraphics[width=9cm]{figs/feedback_page.png}
    \caption{User feedback page}
    \label{fig:User feedback page}
    \end{figure}
When the submit button is clicked, the feedback will automatically be relayed in email form to website developers.


\subsection{User profile options}
In the user profile section, users have the flexibility to make adjustments to several key aspects of their accounts to ensure a personalized experience. They are able to modify the birthday (in case of an initial mistake), update the username to reflect a new online identity, and change the profile picture to express their personality. To find user profile options, users must look at the top right of the website with the profile icon, and click “Edit” in order to end up on the profile screen.
    \begin{figure}[H]
    \centering
    \includegraphics[width=9cm]{figs/profile_page.png}
    \caption{Profile page}
    \label{fig:Profile page}
    \end{figure}

\section*{Change password}
The user can change their passwords as many times as they want. However, passwords cannot be repeated. They must also follow the original protocols regarding the required elements of passwords. Upon changing the password, a confirmation email will be sent to the user.

\section*{Change birthday}
Upon initially signing up for our web applications, users can commonly mistakenly put down the wrong birthday. However our feature allows users to correct that mistake. This feature can be used infinite times.

\section*{Change username}
Within the user profile, users can retain the ability to modify their username to better suit their  identity within Habitat Connect. To enhance freedom of expression, users will have limited restrictions upon choosing their usernames. Restrictions include:
\begin{itemize}
    \item \texttt{Username length:} string of maximum 11 characters
    \item \texttt{Special characters:} 1 special character allowed
    \item \texttt{Duplication: } string cannot be identical to another username in use
\end{itemize}

\section*{Change profile picture}
If users want to further showcase their personality, they can add/edit their profile picture so that other users can see. There will be no restrictions for this feature, and users can change their profile pictures as much as they want.



% REFERENCES
\begin{thebibliography}{00}
\bibitem{b1} Facebook for Developers, "WhatsApp Business API - Documentation," Facebook for Developers. [Online]. Available: \url{https://developers.facebook.com/docs/whatsapp/}\\
\bibitem{b2} "KakaoTalk - Services," Kakao Corporation. [Online]. Available: \url{https://www.kakaocorp.com/page/service/service/KakaoTalk?lang=en}\\
\bibitem{b3} "GitHub - Discord," GitHub. [Online]. Available: \url{https://github.com/discord}\\
\bibitem{b4} "Google Calendar API Documentation," Google Developers. [Online]. Available: \url{https://developers.google.com/calendar}\\
\bibitem{b5} "CI Hackathon July 2020 - GitHub Repository," GitHub. [Online]. Available: \url{https://github.com/maliahavlicek/ci\_hackathon\_july\_2020}\\
\bibitem{b6} "Zendesk - GitHub Repository," GitHub. [Online]. Available: \url{https://github.com/zendesk}\\
\bibitem{b7} "Svelte - GitHub Repository," GitHub. [Online]. Available: \url{https://github.com/sveltejs/svelte}\\
\bibitem{b8} "Node.js - GitHub Repository," GitHub. [Online]. Available: \url{https://github.com/nodejs}\\
\bibitem{b9} "Bootstrap - GitHub Repository," GitHub. [Online]. Available: \url{https://github.com/twbs/bootstrap}\\
\bibitem{b10} "Axios - GitHub Repository," GitHub. [Online]. Available: \url{https://github.com/axios/axios}\\
\bibitem{b11} "MongoDB," MongoDB. [Online]. Available: \url{https://www.mongodb.com}\\
\bibitem{b12} "Visual Studio Code - GitHub Repository," GitHub. [Online]. Available: \url{https://github.com/microsoft/vscode}\\
\bibitem{b13} "Git," Official Git Website. [Online]. Available: \url{https://git-scm.com/}\\
\bibitem{b14} "Overleaf - Documentation," Overleaf. [Online]. Available: \url{https://www.overleaf.com/learn}\\
\bibitem{b15} "Moqups - Online Mockups, Wireframes \& Design Tool," Moqups. [Online]. Available: \url{https://moqups.com/}\\
\bibitem{b16} "JavaScript Documentation," Mozilla Developer Network (MDN). [Online]. Available: \url{https://developer.mozilla.org/en-US/docs/Web/JavaScript}\\
\bibitem{b17} "HTML Documentation," Mozilla Developer Network (MDN). [Online]. Available: \url{https://developer.mozilla.org/en-US/docs/Web/HTML}\\
\bibitem{b18} "CSS Documentation," Mozilla Developer Network (MDN). [Online]. Available: \url{https://developer.mozilla.org/en-US/docs/Web/CSS]}\\
\bibitem{b19} "General Python FAQ," Python.org. [Online]. Available: \url{https://docs.python.org/3/faq/general.html}\\
\end{thebibliography}


\end{document}
